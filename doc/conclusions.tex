%!TEX root =  tfg.tex
\chapter{Final conclusions}

\begin{abstract}
In this final chapter we are going to present our conclusions of the project, recapitulating from the beginning what we have done and analyzing whether this has been a successful project or not. In addition, some ideas will be given to improve the project for the future.
\end{abstract}

\section{Personal opinion}

To conclude this project, we are going to review everything developed in this work and whether it corresponds to the initial idea we had. The objective of this project was to apply the techniques and algorithms of deep learning for the prediction and classification of moles suspected to be possibly carcinogenic. In addition, a small mobile application was to be developed as a practical example of the use of this classifying model. 

In the first place, a state of the art has been realized through the systematic analysis of the tools, architectures and techniques used by the companies and groups of research companies, which has led us to have a theoretical basis. We have also analysed a set of articles that provided us with some information on deep learning in general, and some of them also showed us the behaviour of other techniques in our field of study as well as a formation of the technologies to carry out this work in a successful way.

Next, we developed a set of tools to support our analysis, although due to its generic nature, we adapted it for use in any automatic learning project.  We have obtained the set of images from ISIC, we have used pre-processing techniques and data augmentation to make our results as optimal and high as possible. We also divided the data into a training set and a test set.

We have carried out different models comparing them among them obtaining a model whose result surpasses what was expected at the beginning of the project. In this way, the effectiveness of the application of deep learning techniques for the analysis of medical images was verified.

The model was deployed on a server and finally we made a mobile application with some of the latest technologies to demonstrate that the use of these technologies are not only focused on the academic world but also in the real world, creating products that can be useful for people every day.

As a personal opinion it is worth mentioning that this project has been fascinating for me, in spite of the difficulties that it presented since I did not have the previous knowledge to use the technologies and to apply them to a field like the health which I am not specialized. I consider that I have faced the difficulties of the project, besides writing the documentation in English that is not my native language to obtain a project as professional as possible. 



\section{Future works}

At the end of this project, different possible ideas emerge that will allow us to increase the results and accuracy of our model. 

One of them contemplates, the architecture of a more complex software system to save the images, to treat them so that they had more quality. These images could come from the same application or from specialized centers. It would also be interesting to store meta information such as age, sex, birthplace, etc.. So we could create predictive models to know who and where are the people most likely to have skin cancer and from there discover the reason.

With respect to the mobile application section, it can be improved considerably as applications that currently exist in which the patient and the mole are monitored to see how size and colour change and to see if the risk of skin cancer increases or not.
